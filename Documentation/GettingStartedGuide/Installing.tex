\section{Where to get \CNAME}\label{sec:installing}

See \url{http://www.niwa.co.nz/} for information about \CNAME . The \CNAME\ source code is available on GitHub at \url{https://github.com/NIWAFisheriesModelling/CASAL2}\index{github}.

The \CNAME\ bundle, which includes the \CNAME\ executable, user manuals, example models, the \textbf{R} library, and other information, can be downloaded at

\begin{itemize}
	\item GNU/Linux, \url{ftp://ftp.niwa.co.nz/Casal2/linux/Casal2.tar.gz}
	\item Microsoft Windows, \url{ftp://ftp.niwa.co.nz/Casal2/windows/Casal2.zip}
\end{itemize}

For both 64-bit Linux and Microsoft Windows, only the executable file \texttt{casal2} or \texttt{casal2.exe}, respectively, is required to run \CNAME\ with non-automatic differentiation minimisers. To use the automatic differentiation minimisers, the \texttt{.so} or \texttt{.dll} file must be in the same folder as the executable \CNAME\ file.

On Linux:  if the command \texttt{casal2 -h} does not run, then copy the file \texttt{casal2\_release.so} to \path{/usr/local/lib/} (or to another subdirectory in your \texttt{PATH}, e.g., \path{/home/[username]/bin}).


